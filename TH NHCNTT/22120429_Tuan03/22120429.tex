\documentclass{article}
\usepackage{amsmath}
\usepackage{graphicx}
\usepackage{hyperref}
\usepackage{xcolor}\hypersetup{linkbordercolor=blue}


\title{Hello Latex!}
\author{Hoang Quoc Viet}
\date{\today}

\graphicspath{{imgs/}}

\begin{document}
\maketitle

\section{Getting Started}
\textbf{Hello Latex!} Today I am learning \LaTeX. \LaTeX is a great program for writing math. I can test Vietnamese language ``Hoc vui thôi''.
\\
\indent I can write in line math such as $a^2 + b^2 = c^2$. I can also give equations their own space:
\begin{equation}
    \gamma^2 + \phi^2 = \omega^2
\end{equation}

``Maxwell’s equations'' are named for James Clark Maxwell and are as follow:

\begin{equation} \label{eq:Gauss2}
\vec{\nabla} \cdot \vec{E} =\frac{\rho}{\epsilon_{0}}  \qquad { Gauss's Law }
\end{equation}
\begin{equation} \label{eq:Gauss3}
\vec{\nabla} \cdot \vec{B} = 0 \qquad { Gauss's Law for Magnetism }
\end{equation}  


Equations \ref{eq:Gauss2} and \ref{eq:Gauss3} are some of the most important in Physics.

\section{What about Matrix Equations?}
$$
    \begin{pmatrix}
        a_{11} & a_{12} & \cdots & a_{1n}\\
        a_{21} & a_{22} & \cdots & a_{2n}\\
        \vdots & \vdots & \ddots & \vdots\\
        a_{n1} & a_{n2} & \cdots & a_{nn}
    \end{pmatrix}
    \begin{bmatrix}
        v_1\\
        v_2\\
        \vdots\\
        v_n
    \end{bmatrix}
    =
    \begin{matrix}
        w_1\\
        w_2\\
        \vdots\\
        w_n
    \end{matrix}
$$

\section{Tables and Figures}
\text Creating table is not unlike creating matrix:
\begin{table}
    \caption{This is a table that shows how to create different lines as well as different justifications}
    \label{tab:my_label}
    \centering
    \begin{tabular}{|l||c|c|r|} 
    \hline
     x & 1 & 2 & 3 \\
    \hline\
    f(x) & 4 & 8 & 12 \\
    f(x) & 4 & 8 & 123 \\
    \hline
\end{tabular}
\end{table}

\begin{figure}[htp]
    \centering
    \includegraphics[width=12cm]{imgs/img-1.jpg}
    \caption{HCM University of science}
    \label{fig:HCM University of science}
\end{figure}

\section{Bibliography}
You will probably want references in your document so that you can cite articleslike ~\cite{frenkel_temperature_2012, frenkel_fine_2013, frenkel_optical_2013}

\bibliographystyle{plain}
\bibliography{bibl}
 
\end{document}
